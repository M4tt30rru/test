\section{Conclusions}
\label{Conclusions}
In this work we present a longitudinal analysis on the evolution of of a large software system, 
focusing specifically on its defectiveness.
Through a complex network approach we were able to study the structure of the system, 
by retrieving the community structure of the associated 
network. After having associated to the software network classes, their corresponding bugs, we performed 
a topological analysis.
We found that there is a power law relationship between the maximum values of the clustering coefficient, 
the average bug number 
and the division in communities of the software network. 
This lead to a linear relationship between the maximum values of clustering coefficient and average bug number.
We show that this relationship can in principle be used as a predictor for the maximum value of the average 
bug number in future releases. 
