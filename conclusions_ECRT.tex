\section{Conclusions}
\label{Conclusions}
In this work we presented a longitudinal analysis on the evolution of a large software system with a focus on software defectiveness.
Through a complex network approach we were able to study the structure of 
the system by retrieving the community structure of the associated 
network. After retrieving the number of defects associated to the 
software network classes, we performed a topological analysis detecting the 
community structure. 
We found a power law relationship between the maximum values of the clustering coefficient, the average bug number and the division in communities of the software network. This lead to a linear relationship between the maximum values of clustering coefficient and of average bug number.
We show that such relationship can in principle be used as a predictor for the maximum value of the average bug number in future releases. 
