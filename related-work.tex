\section{Related Work}
\label{sec:related work}

\textbf{DAL PRIMO WETSOM}

%One of the key concepts of OO paradigm is the class, which provides 
%the definition of objects, single units where operative code resides, 
%interacting with other objects in the system by the exchange of messages. 
%The exchange of messages at run time is determined 
%at the source code level by the relationships among the 
%different classes, 
%such as inheritance, composition, aggregation,
%association, dependence and so on. 
Many software systems have reached such a huge dimension that it looks sensible to treat them as 
complex networks \cite{Focardi:2001}, \cite{Myers:2003}, \cite{Turnu:2011}. 
In particular, software networks made of thousands of nodes 
show properties typical of complex networks \cite{Barabasi:2000}, like 
a power law distribution of the node degree, scale free properties \cite{Song:2005}, fractal and self-similar features
\cite{Turnu:2012}, the small world property, power law scaling for 
bug distribution \cite{TSE:2011}, for refactored classes \cite{CSMR:2012}, 
and so on. 
In the field of software, class diagrams \cite{Valverde:2002, Valverde:2003}, 
the software 
collaboration graph \cite{Myers:2003}, the package dependencies networks \cite{Challet:2004}, 
have been found to belong to the complex networks category.
In our case networks nodes are Netbeans classes, and network edges 
are class dependencies.
%\input{related_2}
%Software systems are designed to be highly evolvable \cite{Myers:2003}, in order to be modified to meet new requirements along 
%the time and this evolvability is acquired by decoupling the different subsystems, to
%prevent the risk of side-effects when introducing new functionalities.
%Several authors applied network analysis to the study of software complex systems to gain major knowledge on the 
%structure and behavior of complex systems and also to gather significative information in order to better address the issue of
%developers' effort at each stage of software development. 
%\cite{Wen:2007, Li:2008}.
%[16, 27, 32, 34]. 
In \cite{Wen:2007} it has been shown that the knowledge of scale-free properties of software networks
could be useful to reduce the time devoted to maintenance. 
\v{S}ubelj and Bajek, after analyzing different Java softwares using a community detection algorithm, find that the software systems analyzed 
exhibit a significant community structure that doesn't match the packages structure imposed by the designer \cite{Subelj:2011}. 
\\
The community structure is 
one of the properties that recently grabbed researchers' attention. 
Inside a network a community is a set of vertices between which there is a high density of connections. 
On the contrary between the communities connections are more sparse. The community structure of a network is its 
division in subgroups or communities \cite{NG:2004}.
Elements belonging to the same community are more likely to share the same behavior or properties, or represent a functional unit. 
This leads to many concrete applications of community detection in different fields (from marketing research to software development). 
%Modern software systems are structured in subsystems, generally organized in a hierarchical manner, 
%where small units cooperate with each other to carry out 
%complex tasks through collaboration and information interchange. 
%For this reason, the community structure analysis applied to software networks can give some insights on the structure
%and functionalities of the software.
%Software engineering practices emphasize the decomposition of complex tasks in smaller ones, to encourage code reuse 
%and agile development \cite{IJSEKE:2012}. 
%Specifically, studying the community structure of a complex network can help to understand how a software system is 
%organized in functional modules. 
%Many authors analyzed software systems developed in different programming languages and styles and find evidences that 
%they show scale-free properties (see, for example, \cite{Wen:2007}). 
%[18, 27, 35]. 
%Others \cite{Li:2008}, after having analyzed different java softwares using a community detection algorithm, find that the software systems analyzed 
%exhibit a significant community structure that doesn't match the packages structure imposed by the designer. 
\\
Community detection is traditionally addressed using techniques like hierarchical clustering and partitional clustering 
\cite{Fortunato:2010} and it faces different problems. One of the most important is finding algorithms that allow network
scalability up to different orders of magnitude 
(from thousands to millions of vertices) to be analyzed in a reasonable amount of time. 
\\
Newman \textit{et al.} proposed several algorithms for community detection
\cite{NG:2004, NFast:2004, Clauset:2004} using different approaches and dealing with the problem of the computational burden. 
Moreover, in \cite{NG:2004} Newman and Girvan introduced a quality function called \textit{modularity}, 
to evaluate how good is a network partition in communities. 
%si parla della modularità, conviene introdurre qui la definizione
In our case study, the analysis of the community structure of the software 
network can help to understand how the software system is organized 
in functional modules. 
\\
In this paper we analyze the community structure 
of the 56 Netbeans subprojects and the community 
properties of each subproject in order to recover information about 
the software structure and quality. We compute four  
typical metrics for the communities, namely the modularity, 
the clustering coefficient, the mean degree and the average path length, 
and compare these metrics to data extracted from the 
software subprojects, such as the number of defects or the 
number of packages. 
To accomplish this comparison we use the research questions approach, 
formulating three questions: 
\begin{itemize}

\item \textbf{RQ1}: \textit{Are there correlations between the 
community structure and software defectiveness?} 

\item  \textbf{RQ2}: \textit{Are there correlations between the community 
metrics and software defectiveness?}

\item  \textbf{RQ3}: \textit{Do the software networks analyzed present a community structure that matches the package
structure devised by developers? 
%If this circumstance occurs, to which extent this matching is present? 
%Is there a quantitative measure of the matching or mismatching?
}
\end{itemize}
The answers to these research questions will be discussed after the analysis 
of the results. 


