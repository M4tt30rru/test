\section{Methodology}
\label{Methodology}

In this work we aim at analysing the structure of a software system using its associated software network.
In order to build the associated software network we parsed software's source code, retrieved from the corresponding Software Control Managers (SCM). 
During this procedure, we associate network nodes to classes and network edges 
to the various relationships between classes (inheritance, composition, etc.). 
We consider the number of defects (bugs) as a main indicator of software quality. 
To exploit this we collected data about the bugs of a software system by 
mining its Bug Tracking Systems (BTS). 
Bugzilla is the BTS adopted by Eclipse, where defects are tagged with a unique ID number.
Usually an entry in BTS is called with the common term 'Issue', and there is usually 
no information about classes associated to defects.
Usually all the maintenance operations on software systems are reported on Software Configuration Management
(SCM) systems like Concurrent Version Systems\footnote{CVS, http://www.nongnu.org/cvs/.},
Git and Subversion. 
However, it is not possible to distinguish between bug fixings or
other actions such as enhancements, since all maintenance operations on software systems are recorded as "commit 
operations".
To obtain a correct mapping between Issue(s) and the related 
Java files (CUs), 
we analyzed the SCM log messages, to identify commits associated to
maintenance operations where Issues are fixed. 
If a maintenance operation is done on a file to address a defect, 
we consider the CU as affected by that defect. \\
We first analyzed the text of commit messages, looking for Issue-IDs.
Unfortunately, every positive integer number is a potential Issue-ID. However, sometimes
numbers that refer to maintenance operations are not related to Issue-ID resolution, but, for example,
to branching, data, number of release, copyright updating, and so on. 
To avoid wrong mappings between Issue-IDs and CUs, we applied the following rules:
\begin{itemize}
\item In each release, a CU can be affected only by Issues which are referred to in the BTS as belonging to the same release.
\item All IDs not filtered out are considered Issues and associated to the addition or modification of one or more CUs, as reported in the commit logs. 
\item When assigning  defects to classes in the corresponding CUs, since there were 
few CUs containing more than one class, we decided to assign all the defects to the biggest 
class of those CUs.
\end{itemize}

This method might not completely address the problems 
of the mapping between defects and CUs \cite{Ayari:2007}. 
In any case we checked manually 10$\%$ of randomly chosen CU-defect(s) associations for intermediate releases
and every CU-defect association for 
3 sub-projects without finding any error. A bias may still remain 
due to lack of information on SCM \cite{Ayari:2007}. 
The subset of Issues satisfying the conditions as in Eaddy et al. is the Bug-metric \cite{Eaddy:2008}. 
Of course there are chances for wrong assignments to happen for 
some classes, but since the average number of defects for class is very low,
the number of wrong assignments in the entire system, considering also 
CUs with one class, is very limited. 
\\
At the end of this process we obtained a network where 
to each node is associated the number of bugs of the corresponding class. 

We collected the source code and analysed 5 releases of Eclipse, whose main feature are 
presented in Table \ref{tab:Eclipse}. 

\begin{table}[h]
\begin{center}
\begin{tabular}{|l|c|c|c|c|c|}
\hline 
\textbf{Release} &\textbf{  2.1} & \textbf{ 3.0} &  \textbf{3.1 }&  \textbf{3.2} &  \textbf{3.3
} \\
Size & 8257 & 11406 & 13413 & 16013 & 17517 \\
 
Sub-Projects n.& 49 & 66 & 70 & 86 & 104 \\ 

N. of defects & 47788 & 59804 & 69900 & 80149 & 95337  \\ \hline


\end{tabular}
\end{center}
\caption{Main features of the analysed releases of Eclipse (EC): size (number of classes), 
number of sub-projects (sub-networks), and total number of defects.}
\label{tab:Eclipse}

\end{table}

Each release is structured in almost independent sub-projects. The total number 
of sub-projects analysed amounts at 375, with more than 60000 nodes (classes) 
and more than 350000 defects.% 170623

We detected the associated community structure using the algorithm devised by 
Clauset et al. \cite{Clauset:2004}. 
This is an agglomerative clustering algorithm that performs a greedy optimization 
of the Modularity (Q) \cite{Newman:2003}. 
At the end we retrieved the number of communities in which the network is structured, 
the corresponding value Q of the modularity, and the nodes associated to each community. 
We performed the computation of the clustering coefficient using the implementation 
included in the IGraph software 
\cite{igraph}. 
To study the system's evolution we used the following approach. We first 
carried out the analysis for each release, and then we assembled together 
different releases, according to a temporal evolution. 
More precisely, for the 5 releases of our dataset, we
studied the evolution of the system by cumulating the first and 
the second releases in a single set, then adding the third release 
to this first set to obtain a second set and so on. 
This way we were able to make predictions about the next release 
starting from those cumulated in the previous assembly.

